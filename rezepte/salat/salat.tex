%! Author = jakob
%! Date = 18.03.22

\chapter[Salat]{Salat}

\section{Warmer Linsensalt}
\subsection{Zutaten}
\begin{supertabular}{\zutatenspalten}
    1&Zwiebel\\
    200g&Linsen, am besten eignen sich dunkel grüne\\
    150g&Speck\\
    1 kleiner Bund&Schnittlauch, fein geschnitten\\
    1&Schalotte, fein geschnitten\\
    \subzutat{Dressing}\\
    1TL&Dijon-Senf\\
    1EL&Rotweinessig\\
    2EL&Walnussöl\\
    &Salz und Pfeffer\\
\end{supertabular}
\subsection{Zubereitung}
Die Zwiebel schälen. Die Linsen unter kaltem Wasser waschen. Die Linsen in einen großen Topf geben und knapp 2 Liter Wasser hinzugeben. Das Wasser leicht salzen und die Zwiebel hinzugeben. Das Wasser aufkochen lassen und dann die Hitze reduzieren. Den Deckel auf den Topf geben und abhängig von der Wahl der Linsen kochen lassen. Die Linsen sollten schön zart sein, aber noch nicht zerfallen.\\
Die Schwarte vom Speck entfernen und in dünne Streifen schneiden.Den Speck in einer Pfanne schön karamellisieren. Das Dressing mischen. Alle Zutaten vermischen und servieren. Der Salat schmeckt am besten, wenn er mindestens 15 Minuten lang gezogen hat. Dementsprechend ist es sinnvoll, den Speck erst kurz vor dem Essen anzubraten. Es ist schön, wenn mindestens der Speck noch warm ist.
\newpage

\section{Bratkartoffelsalat}
\subsection{Zutaten}
\begin{supertabular}{\zutatenspalten}
    750g&Kartoffeln\\
    100g&Zuckerschoten\\
    3-4&Zuckerschoten\\
    1 Bund&Radieschen\\
    1Bund&Rucola\\
    1 Bund&Dill\\
    &Salz und Pfeffer\\
    \subzutat{Dressing}\\
    4EL&Gemüsebrühe \jref{sec:gemuesebruehe}\\
    2EL&Dijonsenf\\
    2EL&Apfelessig\\
    1EL&Zitronensaft\\
    2EL&Reissirup\\
    1EL&Olivenöl\\
    &Salz und Pfeffer\\
    \subzutat{Zum Anrichten}\\
    einige&Schnittlauchblüten\\
\end{supertabular}
\subsection{Zubereitung}
Die Kartoffeln gut wachen oder schälen und im Ganzen im Wasser 10-20 Minuten (je nach Größe) gar kochen. Die Kollen abkühlen lassen, halbieren oder vierteln. und in Olivenöl in einer Pfanne in ca. 15 Minuten knusprig anbraten, dabei zwischendurch wenden. Mit Salz, Pfeffer und Muskatnuss würzen und abkühlen lassen.\\
In der Zwischenzeit die Zuckerschoten falls nötig entfädeln, in sprudelnd kochendem Wasser blanchieren und in Eiswasser abschrecken, damit sie schön grün und knackig bleiben. Die Zuckerschoten abtropfen lassen und nach Belieben halbieren. Frühlingszwiebeln und Radieschen in feine Scheiben schneiden und zusammen mit Zuckerschoten und Rucola unter die Kartoffeln heben. Den Dill ohne dicke Stängel klein hacken.\\
Für das Dressing alle Zutaten bis auf das Olivenöl vermischen. Das Öl zugeben und gut unterschlagen. Das Dressing mit Salz und Pfeffer abschmecken und unter die Kartoffeln heben. Den Salz mit Dill bestreuen und (falls verwendet) mit Schnittlauchblüten garnieren.
\newpage

\section{Tomaten-Melonen-Salat}
\subsection{Zutaten}
\begin{supertabular}{\zutatenspalten}
    750g&Wassermelone (ohne Schale)\\
    10&große bunte Tomaten\\
    1&Schalotte, fein gehackt\\
    1&Knoblauchzehe, fein gehackt\\
    1EL&Olivenöl\\
    1EL&Zitronensaft\\
    1EL&Reissirup\\
    $\frac{Bund}{2}$&Basilikum\\
    200g&Mozzarella\\
    &Salz und Pfeffer\\
\end{supertabular}
\subsection{Zubereitung}
Die Melone ohne Schale in kleine Dreiecke schneiden. Die Tomaten halbieren, den Stielansatz entfernen und das Fruchtfleisch in mundgerechte Stücke schneiden. Melone und Tomaten in einer großen Schüssel mischen.\\
Schalotte und Knoblauch in einer Pfanne anschwitzen. Zitronensaft und Reissirup einrühren, das warme Dressing mit Salz und Pfeffer würzen und über dem Salat verteilen. Die Basilikumblätter und den Mozzarella zerrupfen und beides unter den Salat mischen.
\newpage

\section{klassischer Kartoffelsalat}
\subsection{Zutaten}
\begin{supertabular}{\zutatenspalten}
    1kg&festkochende Kartoffeln\\
    2&Zwiebeln\\
    \subzutat{Dressing}\\
    250ml&heiße Brühe\\
    4EL&Öl\\
    4-6EL&Rahm oder Joghurt\\
    2-3EL&Zitronensaft oder Essig\\
    1 Prise&Zucker\\
    &Salz\\
    \subzutat{Nach Belieben}\\
    &Kräuter, fein gehackt\\
    3&Gewürzgurzen\\
    1&frische Gurke\\
    3&rohe Äpfel\\
    $\frac{1}{2}$&Sellerieknolle, gekocht\\
    &magere Fleisch-, Wurst- oder Schinkenreste\\
    2&Heringsfilets\\
    &Mayonnaise, für das Dressing\\
\end{supertabular}
\subsection{Zubereitung}
Salatkartoffeln frisch dämpfen, schälen, etwas abkühlen lassen, in Scheiben schneiden, mit fein geschnittenen Zwiebeln mischen; die noch warmen Kartoffeln mit heißer, gut abgeschmeckter Marinade ohne Öl anmachen, vorsichtig mischen, abschmecken, gut durchziehen lassen, zuletzt Öl oder angeröstete kleine Speckwürfel untermischen, vor dem Anrichten nochmals abschmecken. Bei Zugabe von Mayonnaise Kartoffelscheiben mit heißer Marinade ohne Öl gut durchziehen lassen, erst vor dem Abrichten mit Mayonnaise mischen.\\
Kartoffelsalat kann durch Zugabe verschiedener Geschmackszutaten sehr abwechslungsreich gestaltet und abgewandelt werden: Angegebene Verän-derungszutaten je nach Wahl in kleine Würfel oder Streifen schneiden, unter den Kartoffelsalat mischen, gut durchziehen lassen. Frisch Gurken jeweils schälen, hobeln oder in feine Scheiben schneiden, kurz vor dem Anrichten untermischen. Die verschiedenen Abwandlungszutaten können einzeln oder gemischt verwendet werden; in letzterem Fall Kartoffelmenge etwas verringern.\\
\newpage