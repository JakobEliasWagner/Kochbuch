%! Author = jakob
%! Date = 18.03.22
%! brot


\chapter{Brot}


\section{Mildes Sauerteigbrot}
\begin{tabular}{ll}
    Zubereitung am Backtag: & 1 Stunde      \\
    Zubereitung gesamt:     & 14-18 Stunden
\end{tabular}\\

\paragraph{}
\begin{tabular}{ll}
    \textbf{Weizensauerteig} \\
    75g  & Weizenmehl 550       \\
    75g  & Wasser (40$^\circ$C) \\
    \textbf{Hauptteig} \\
    425g & Weizenmehl 550       \\
    275g & Wasser (25$^\circ$C) \\
    11g  & Salz                 \\
\end{tabular}\\

\paragraph{}
Die Sauerteigzutaten vermischen und 2 Stunden bei 27$^\circ$C reifen lassen.\\
Die Teigzutaten 5 Minuten auf niedrigster Stufe und weitere 10 Minuten auf zweiter Stufe zu einem feuchten, mittelfesten Teig kneten (Teigtemperatur ca. 27$^\circ$C).\\
3 Stunden bei 27$^\circ$C ruhen lassen. Dabei alle 30 Minuten dehnen und falten.\\
Einen straffen runden Laib formen und mit Schluss nach oben für 8-12 Stunden bei 6-8$^\circ$C im gut bemehlten Gärkorb zu Reife stellen.\\
Den Laib einschneiden und mit Schluss nach unten im auf 250$^\circ$C aufgeheizten Gusseisentopf (wahlweise mit Dampf auf Backstein) 45 Minuten fallend auf 220$^\circ$C backen.
\newpage


\section{Mediterranes Weizensauerteigbrot}
\begin{tabular}{ll}
    Zubereitung am Backtag: & 6,5 Stunden \\
    Zubereitung gesamt:     & 20 Stunden
\end{tabular}\\

\paragraph{}
\begin{tabular}{ll}
    \textbf{Sauerteig} \\
    82g  & Weizenmehl 812       \\
    44g  & Wasser (50$^\circ$C) \\
    82g  & Anstellgut           \\
    \textbf{Autolyseteig} \\
    110g & Hartweizenmehl       \\
    138g & Weizenmehl 1050      \\
    330g & Wasser (50$^\circ$C) \\
    \textbf{Hauptteig} \\
    & Sauerteig            \\
    & Autolyseteig         \\
    44g  & Wasser (50$^\circ$C) \\
    12g  & Salz                 \\
    11g  & Olivenöl             \\
\end{tabular}\\

\paragraph{}
Die Sauerteigzutaten von Hand zu einem festen Teig mischen und 2-3 Stunden bei ca. 27$^\circ$C reifen lassen (keine Vollreife). Anschließend für 10-12 Stunden bei 5$^\circ$C nachreifen lassen.\\
Die Autolysezutaten vermengen und 60 Minuten verquellen lassen.\\
Die Hauptteigzutaten vermischen (Teigtemperatur ca. 28$^\circ$C).\\
3 Stunden Gare bei ca. 27$^\circ$C. Dabei in den ersten 2 Stunden alle 20 Minuten dehnen und falten.\\
Den Teig schonend (Gas erhaltend) rundwirken und 60 Minuten mit Schluss nach unten im Gärkorb bei ca. 27$^\circ$C zur Gare stellen.\\
Bei 280$^\circ$C (alternativ 250$^\circ$C) fallend auf 220$^\circ$C (230$^\circ$C) 45 Minuten backen. Nach 2 Minuten kräftig schwaden. Schwaden nach 10 Minuten wieder ablassen.
\newpage


\section{Berliner Roggenmischbrot}
\begin{tabular}{ll}
    Zubereitung am Backtag: & 3,5 Stunden \\
    Zubereitung gesamt:     & 19 Stunden
\end{tabular}\\

\paragraph{}
\begin{tabular}{ll}
    \textbf{Sauerteig} \\
    200g & Alpenroggen (alternativ: Roggenmehl 1370)    \\
    200g & Wasser (45$^\circ$C)                         \\
    40g  & Anstellgut                                   \\
    4g   & Salz                                         \\
    \textbf{Quellstück} \\
    50g  & Altbrot (getrocknet, gemahlen)               \\
    150g & Wasser (siedend)                             \\
    7g   & Salz                                         \\
    \textbf{Hauptteig} \\
    & Sauerteig                                    \\
    & Quellstück                                   \\
    150g & Alpenroggen (alternativ Roggenmehl 997/1150) \\
    150g & Tipo 0 (alternativ: Weizenmehl 812/1050)     \\
    2g   & Schabzigerklee                               \\
    150g & Wasser (45$^\circ$C)                         \\
\end{tabular}\\

\paragraph{}
Die Sauerteigzutaten mischen und 12-16 Stunden bei 20-22$^\circ$C reifen lassen.\\
Altbrot und Salz mit Wasser mischen, mit Folie direkt auf der Oberfläche abdecken und 2-12 Stunden kühl lagern.\\
Alle Zutaten 8 Minuten auf niedrigster Stufe und weitere 2 Minuten auf zweiter Stufe zu einem feuchten, bindigen Teig kneten (Teigtemperatur ca. 28-30$^\circ$C).\\
30 Minuten Teigruhe bei ca. 24$^\circ$C.\\
Den Teig rundwirken und mit offenem Schluss nach unten in einen bemehlten Gärkorb setzen.
90 Minuten Gare bei ca. 24$^\circ$C.\\
Bei 250$^\circ$C fallend auf 220$^\circ$C mit Schluss nach oben und mit Dampf 45-50 Minuten tiefbraun backen.
\pagebreak


\section{Weißes Brot}
\begin{tabular}{ll}
    Zubereitung am Backtag: & 4 Stunden     \\
    Zubereitung gesamt:     & 48-72 Stunden
\end{tabular}\\\paragraph*{}
\begin{tabular}{ll}
    \textbf{Vorteig} \\
    285g & Weizenmehl 550                 \\
    15g  & Roggenmehl 997                 \\
    87g  & Milch (3,5\% Fett, 5$^\circ$C) \\
    87g  & Wasser (kalt)                  \\
    6g   & Salz                           \\
    6g   & Frischhefe                     \\
    \textbf{Hauptteig} \\
    & Vorteig                        \\
    285g & Weizenmehl 550                 \\
    15g  & Roggenmehl 997                 \\
    87g  & Milch (3,5\% Fett, 5$^\circ$C) \\
    87g  & Wasser (70$^\circ$C)           \\
    6g   & Salz                           \\
\end{tabular}\\\paragraph*{}
Die Vorteigzutaten 5 Minuten auf niedrigster Stufe und 5 Minuten auf zweiter Stufe zu einem mittelfesten Teig kneten. 2-3 Tage bei 5$^\circ$C im Kühlschrank lagern.\\
Sämtliche Hauptteigzutaten 5 Minuten auf niedrigster Stufe und 1-2 Minuten auf zweiter Stufe kneten (Teigtemperatur ca. 25$^\circ$C).\\
Den Teig 100 Minuten bei Raumtemperatur (ca. 20$^\circ$C) ruhen lassen.\\
Den Teig rund- und langwirken und mit Schluss nach oben 90 Minuten in einem mit Kartoffelstärke ausgestaubten Gärkorb bei Raumtemperatur reifen lassen.\\
Den Teigling aus dem Korb stürzen, dreimal quer und tief einschneiden und mit heißem Wasser abstreichen.\\
Bei 230$^\circ$C fallend auf 210$^\circ$C 35-40 Minuten mit viel Dampf backen.
\pagebreak


\section{Fladenbrot nach türkischer Art}
\begin{tabular}{ll}
    Zubereitung am Backtag: & $\approx 4,5 Stunden$ \\
    Zubereitung gesamt:     & $\approx 17 Stunden$
\end{tabular}\\\paragraph*{}
\begin{tabular}{ll}
    \textbf{Vorteig A} \\
    100g & Weizenmehl 550                    \\
    100g & Milch (3,5\% Fett, 5$^\circ$C)    \\
    0,1g & Frische Hefe                      \\
    \textbf{Vorteig B} \\
    60g  & Weizenvollkornmehl                \\
    60g  & Joghurt (3,8\% Fett, 5$^\circ$C)  \\
    0,3g & Frischhefe                        \\
    \textbf{Hauptteig}
    & Vorteig A                         \\
    & Vorteig B                         \\
    240g & Weizenmehl 550                    \\
    120g & Wasser (20$^\circ$C)              \\
    16g  & Pflanzenöl                        \\
    9g   & Salz                              \\
    12g  & Zucker oder Flüssigmalz (inaktiv) \\
    3,5g & Frischhefe                        \\
\end{tabular}\\\paragraph*{}
Die jeweiligen Vorteigzutaten mischen und 12 Stunden bei Raumtemperatur (20$^\circ$C) reifen lassen.\\
Alle Hauptteigzutaten 5 Minuten auf niedrigster Stufe und 10 Minuten auf zweiter Stufe kneten (Teigtemperatur ca. 24$^\circ$C).\\
Den Teig 2,5 Stunden bei Raumtemperatur ruhen lassen. Nach 30, 60 und 90 Minuten dehnen und falten.\\
350 g-Stücke abstechen und schonend rund einschlagen.\\
30 Minuten mit Schluss nach oben ruhen lassen.\\
Den Teigling umdrehen und mit den Fingerspitzen bodentief eindrücken, sodass der Teigball flach wird und ihn nochmals für 30 Minuten mit der eingedrückten Seite nach unten (am besten im Leinentuch) ruhen lassen.\\
Den Teigling umdrehen, auf Backpapier setzen, nochmals leicht mit den Fingerspitzen eindrücken, mit Milch abstreichen und mit Kreuzkümmel bestreuen.\\
Bei 250$^\circ$C fallend auf 230$^\circ$C 20 Minuten mit Dampf backen.
Sofort nach dem Backen nochmals mit Milch abstreichen.
\newpage


\section{Baguette mit Kochstück}
\begin{tabular}{ll}
    Zubereitung am Backtag: & $\approx$5 Stunden  \\
    Zubereitung gesamt:     & $\approx$17 Stunden
\end{tabular}\\\paragraph*{}
\begin{tabular}{ll}
    \textbf{Sauerteig} \\
    75g   & Weizenmehl 550                  \\
    75g   & Wasser (50$^\circ$C)            \\
    7,5g  & Anstellgut                      \\
    \textbf{Vorteig} \\
    100g  & Weizenmehl 550                  \\
    100g  & Wasser (kalt)                   \\
    1g    & Frischhefe                      \\
    \textbf{Mehlkochstück} \\
    15g   & Weizenmehl 550                  \\
    75g   & Wasser                          \\
    11g   & Salz                            \\
    \textbf{Autolyseteig} \\
    & Vorteig                         \\
    & Sauerteig                       \\
    & Mehlkochstück                   \\
    310g  & Weizenmehl 550                  \\
    100g  & Wasser (30$^\circ$C)            \\
    \textbf{Hauptteig} \\
    & Autolyseteig                    \\
    5g    & Frischhefe\\
    (20g) & Olivenöl, optional für Focaccia \\
\end{tabular}\\\paragraph*{}
Die Sauerteigzutaten vermischen und 12 Stunden bei Raumtemperatur (ca. 20$^\circ$C) reifen lassen.\\
Die Vorteigzutaten mischen und 10-12 Stunden bei 12-14$^\circ$C reifen lassen (alternativ nur ein Zehntel der Hefemenge verwenden und 12 Stunden bei Raumtemperatur reifen lassen. Dann aber nur 20$^\circ$C warmes Wasser in den Autolyseteig geben).\\
Mehl und Salz mit dem Wasser mischen und unter Rühren rasch aufkochen bis die Masse eindickt. Direkt auf die Oberfläche Klarsichtfolie drücken und auskühlen lassen (bis zu 24 Stunden bei Raumtemperatur lagerfähig).\\
Für den Autolyseteig Mehl, Wasser, Vorteig, Sauerteig und Mehlkochstück mischen und 30 Minuten bei Raumtemperatur ruhen lassen.\\
Hefe (bei Focaccia auch noch Olivenöl) zugeben, 5 Minuten auf niedrigster Stufe und 3 Minuten auf zweiter Stufe kneten (Teigtemperatur ca. 25$^\circ$C).\\
Den Teig 2,5 Stunden bei Raumtemperatur ruhen lassen. In den ersten 90 Minuten alle 30 Minuten falten.\\
300 g-Teiglinge abstechen, zu Zylindern vorformen und 30 Minuten mit Schluss nach oben in Bäckerleinen ruhen lassen.\\
Die Zylinder zu Baguettes formen und nochmals 20-30 Minuten in Leinen reifen lassen.\\
Einschneiden und bei 250$^\circ$C 20 Minuten mit Dampf backen.\\
Für Focaccia die vorgeformten Teiglinge mit Olivenöl beträufeln und mit den Fingerspitzen tief eindrücken, sodass der Teigling flach und länglich wird.\\
30 Minuten ruhen lassen.\\
Einen dünnen Belag nach Wahl (oder erneut Olivenöl) darauf geben und nochmals mit den Fingerspitzen einmassieren.\\
Nochmals 30 Minuten ruhen lassen und anschließend 15-20 Minuten mit Dampf bei 250$^\circ$C backen.\\
\newpage


\section{Dinkel-Baguettes}
\begin{tabular}{ll}
    Zubereitung am Backtag: & $\approx$1,5 Stunden \\
    Zubereitung gesamt:     & $\approx$72 Stunden
\end{tabular}\\\paragraph*{}
\begin{tabular}{ll}
    \textbf{Kochstück} \\
    55g  & Dinkelmehl 630 \\
    275g & Wasser         \\
    \textbf{Autolyseteig} \\
    & Kochstück      \\
    275g & Dinkelmehl 630 \\
    \textbf{Hauptteig} \\
    & Autolye-Teig   \\
    2g   & Frischhefe     \\
    6,5g & Salz           \\
    2g   & Zucker         \\
\end{tabular}\\\paragraph*{}
Mehl mit Wasser verrühren, aufkochen und 2 Minuten lang auf der abkühlenden Herdplatte rühren bis eine zähe Masse entstanden ist. Auskühlen und mind. 4-12 Stunden ruhen lassen.\\
Mehl und Kochstück von Hand homogen vermengen und 1 Stunde ruhen lassen (Autolyse).\\
Die übrigen Zutaten von Hand einarbeiten bis alles gut vermengt ist.\\
60 Minuten Teigruhe bei 24$^\circ$C, dabei alle 20 Minuten falten.\\
48-72 Stunden Gare bei 4-6$^\circ$C im Kühlschrank.\\
3 Teiglinge zu je ca. 190-200 g abstechen und zu Zylindern aufrollen (vorformen).\\
Die Zylinder 15 Minuten in Bäckerleinen ruhen lassen.\\
Die Baguettes formen und 30 Minuten bei ca. 24$^\circ$C mit Schluss nach oben in Bäckerleinen gehen lassen.\\
Die Teiglinge vom Bäckerleinen auf Backpapier oder einen Brotschieber stürzen und mit 3 Schnitten im spitzen Winkel zur Längsachse mit flacher Klinge einschneiden.\\
Bei 250$^\circ$C fallend auf 230$^\circ$C mit Dampf 25 Minuten backen.\\
\newpage


\section{Kartoffelbaguette}
\begin{tabular}{ll}
    Zubereitung am Backtag: & 4 Stunden  \\
    Zubereitung gesamt:     & 16 Stunden
\end{tabular}\\\paragraph*{}
\begin{tabular}{ll}
    \textbf{Vorteig} \\
    100g & Weizenmehl 1050               \\
    100g & Wasser (kalt)                 \\
    \textbf{Autolyseteig (Quellstück)} \\
    400g & Weizenmehl 550                \\
    660g & Kartoffeln (gekocht, gepellt) \\
    \textbf{Hauptteig} \\
    & Vorteig                       \\
    & Autolyseteig                  \\
    4g   & Frischhefe                    \\
    10g  & Salz                          \\
\end{tabular}\\\paragraph*{}
Die Vorteigzutaten vermengen. 1 Stunde bei Raumtemperatur (20$^\circ$C) anspringen lassen, anschließend 11 Stunden bei ca. 9-10$^\circ$C im Kühlschrank lagern. Der Vorteig sollte am Ende sehr von Blasen durchsetzt sein und fruchtig-alkoholisch riechen.\\
Für den Autolyseteig Kartoffeln zerdrücken und gemeinsam mit dem Mehl von Hand zu einem festen, etwas klebenden Teig verrühren (grob, so dass das Mehl gut eingearbeitet ist). 12 Stunden bei ca. 18-20$^\circ$C ruhen lassen.\\
Alle Zutaten für den Hauptteig 6-8 Minuten von Hand zu einem mittelfesten Teig vermengen (alternativ 2 Minuten auf niedrigster Stufe und 2 Minuten auf zweiter Stufe in der Knetmaschine). Der Teig klebt etwas. Teigtemperatur: ca. 19$^\circ$C.\\
3 Stunden Gare bei ca. 24$^\circ$C. In den ersten 2 Stunden alle 30 Minuten intensiv dehnen und falten. Der Teig ist am Ende straff und feucht und von Blasen durchzogen.\\
Drei 430 Gramm-Stücke abstechen, schonend zu ca. 20 cm langen Zylindern einrollen und mit beiden Händen zu ca. 30 cm langen Baguettes ausrollen.\\
Mit Schluss nach oben 30 Minuten bei 24$^\circ$C in Bäckerleinen gehen lassen (knappe Gare).\\
Einschneiden und bei 25$^\circ$C 25 Minuten mit viel Dampf backen.


\section{Kartoffelbaguette}
\begin{tabular}{ll}
    Zubereitung am Backtag: & 4 Stunden  \\
    Zubereitung gesamt:     & 16 Stunden
\end{tabular}\\\paragraph*{}
\begin{tabular}{ll}
    \textbf{Vorteig} \\
    100g & Weizenmehl 1050               \\
    100g & Wasser (kalt)                 \\
    \textbf{Autolyseteig (Quellstück)} \\
    400g & Weizenmehl 550                \\
    660g & Kartoffeln (gekocht, gepellt) \\
    \textbf{Hauptteig} \\
    & Vorteig                       \\
    & Autolyseteig                  \\
    4g   & Frischhefe                    \\
    10g  & Salz                          \\
\end{tabular}\\\paragraph*{}
Die Vorteigzutaten vermengen. 1 Stunde bei Raumtemperatur (20$^\circ$C) anspringen lassen, anschließend 11 Stunden bei ca. 9-10$^\circ$C im Kühlschrank lagern. Der Vorteig sollte am Ende sehr von Blasen durchsetzt sein und fruchtig-alkoholisch riechen.\\
Für den Autolyseteig Kartoffeln zerdrücken und gemeinsam mit dem Mehl von Hand zu einem festen, etwas klebenden Teig verrühren (grob, so dass das Mehl gut eingearbeitet ist). 12 Stunden bei ca. 18-20$^\circ$C ruhen lassen.\\
Alle Zutaten für den Hauptteig 6-8 Minuten von Hand zu einem mittelfesten Teig vermengen (alternativ 2 Minuten auf niedrigster Stufe und 2 Minuten auf zweiter Stufe in der Knetmaschine). Der Teig klebt etwas. Teigtemperatur: ca. 19$^\circ$C .\\
3 Stunden Gare bei ca. 24$^\circ$C. In den ersten 2 Stunden alle 30 Minuten intensiv dehnen und falten. Der Teig ist am Ende straff und feucht und von Blasen durchzogen.\\
Drei 430 Gramm-Stücke abstechen, schonend zu ca. 20 cm langen Zylindern einrollen und mit beiden Händen zu ca. 30 cm langen Baguettes ausrollen.\\
Mit Schluss nach oben 30 Minuten bei 24$^\circ$C in Bäckerleinen gehen lassen (knappe Gare).\\
Einschneiden und bei 25$^\circ$C 25 Minuten mit viel Dampf backen.
\newpage


\section{Brot mit ofengerösteten Kartoffeln}
\begin{tabular}{ll}
    Zubereitung am Backtag: & 5 Stunden  \\
    Zubereitung gesamt:     & 17 Stunden
\end{tabular}\\\paragraph*{}
\begin{tabular}{ll}
    \textbf{Vorteig} \\
    300g & Weizenmehl 550       \\
    195g & Wasser               \\
    4g   & Salz                 \\
    1g   & Trockenhefe          \\
    \textbf{Hauptteig} \\
    & Vorteig              \\
    550g & Weizenmehl 550       \\
    415g & Wasser               \\
    18g  & Salz                 \\
    3g   & Trockenhefe          \\
    350g & Kartoffeln, geröstet
\end{tabular}\\\paragraph*{}
Für den Vorteig alle Zutaten miteinander vermischen und etwa 3-6 Minuten gut verkneten. Bei ca. 21$^\circ$C 12 Stunden gehen lassen. In der Zwischenzeit 350 g Kartoffeln mit Schale in Stücke von 1-2 cm Größe schneiden und im Ofen goldbraun und weich rösten, anschließend abkühlen lassen.\\
Alle Zutaten des Hauptteiges (bis auf den Vorteig) 3 Minuten verkneten und dann den Vorteig zugeben. Den Teig weitere 3-6 Minuten kneten und 1,5 Stunden bei 24$^\circ$C gehen lassen. Nach 45 Minuten 1-3 Mal den Teig falten. Den Teig in zwei Hälften teilen (bei Nutzung der Hälfte der Zutaten entfällt der Schritt natürlich), 10 Minuten abgedeckt entspannen lassen. Anschließend den Laib formen und für stabilen Halt wirken (meine Methode: Teigklumpen auf der bemehlten Arbeitsfläche zwischen beide Hände nehmen, gegen den Uhrzeigersinn drehen und dabei immer etwas Teig nach unten schieben). Nochmals 1,5 Stunden bei 24$^\circ$C gehen lassen und dann in den auf 230$^\circ$C vorgeheizten Backofen schieben. Anfangs eine halbe Tasse Wasser auf den Ofenboden schütten oder das Brot kräftig mit Wasser einsprühen. Nach 40-45 Minuten sollte das Brot dunkelbraun und fertig sein.
\newpage


\section{Brot mit Ofen-gerösteten Kartoffeln}
\begin{tabular}{ll}
    Zubereitung am Backtag:& 3 Stunden\\
\end{tabular}\\\paragraph*{}
\begin{tabular}{ll}
    \textbf{Hauptteig} \\
    680g  & Kartoffeln, mehlige       \\
    16g   & Salz                      \\
    120ml & Kartoffelkochwasser, warm \\
    1EL   & Trockenhefe               \\
    2EL   & Olivenöl                  \\
    665g  & Weizenmehl 550            \\
\end{tabular}\\\paragraph*{}
Die Kartoffeln vierteln und mit der Hälfte vom Salz kochen. Die Backhefe mit dem Kartoffelwasser vermischen und kurz quellen lassen (Dabei sollte das Wasser nicht mehr viel zu warm sein - Die Hefe stirbt ansonsten ab). Die abgekühlten Kartoffeln zerdrücken und mit den restlichen Zutaten verkneten. Der Teig ist am Anfang recht fest und krümelig, wird aber mit dem Kneten immer weicher und homogener.\\
Den Teig bei Raumtemperatur 20-30 Minuten gehen lassen. Anschließend zwei Brote abstechen und Laibe formen. Die Brote dann weitere 30 Minuten, mit Schluss nach unten, gehen lassen.\\
Das Brot dann 50 Minuten bei 190$^\circ$C mit Dampf backen. Nach etwa 20 Minuten Schwaden ablassen.


\section{Kartoffelbaguette}
\begin{tabular}{ll}
    Zubereitung am Backtag: & 4 Stunden  \\
    Zubereitung gesamt:     & 16 Stunden
\end{tabular}\\\paragraph*{}
\begin{tabular}{ll}
    \textbf{Vorteig} \\
    100g & Weizenmehl 1050               \\
    100g & Wasser (kalt)                 \\
    \textbf{Autolyseteig (Quellstück)} \\
    400g & Weizenmehl 550                \\
    660g & Kartoffeln (gekocht, gepellt) \\
    \textbf{Hauptteig} \\
    & Vorteig                       \\
    & Autolyseteig                  \\
    4g   & Frischhefe                    \\
    10g  & Salz                          \\
\end{tabular}\\\paragraph*{}
Die Vorteigzutaten vermengen. 1 Stunde bei Raumtemperatur (20$^\circ$C) anspringen lassen, anschließend 11 Stunden bei ca. 9-10$^\circ$C im Kühlschrank lagern. Der Vorteig sollte am Ende sehr von Blasen durchsetzt sein und fruchtig-alkoholisch riechen.\\
Für den Autolyseteig Kartoffeln zerdrücken und gemeinsam mit dem Mehl von Hand zu einem festen, etwas klebenden Teig verrühren (grob, so dass das Mehl gut eingearbeitet ist). 12 Stunden bei ca. 18-20$^\circ$C ruhen lassen.\\
Alle Zutaten für den Hauptteig 6-8 Minuten von Hand zu einem mittelfesten Teig vermengen (alternativ 2 Minuten auf niedrigster Stufe und 2 Minuten auf zweiter Stufe in der Knetmaschine). Der Teig klebt etwas. Teigtemperatur: ca. 19$^\circ$C .\\
3 Stunden Gare bei ca. 24$^\circ$C. In den ersten 2 Stunden alle 30 Minuten intensiv dehnen und falten. Der Teig ist am Ende straff und feucht und von Blasen durchzogen.\\
Drei 430 Gramm-Stücke abstechen, schonend zu ca. 20 cm langen Zylindern einrollen und mit beiden Händen zu ca. 30 cm langen Baguettes ausrollen.\\
Mit Schluss nach oben 30 Minuten bei 24$^\circ$C in Bäckerleinen gehen lassen (knappe Gare).\\
Einschneiden und bei 25$^\circ$C 25 Minuten mit viel Dampf backen.
\newpage


\section{Brot mit ofengerösteten Kartoffeln - das hier ist nicht so kompliziert}
\begin{tabular}{ll}
    Zubereitung am Backtag: & 5 Stunden  \\
    Zubereitung gesamt:     & 17 Stunden
\end{tabular}\\\paragraph*{}
\begin{tabular}{ll}
    \textbf{Vorteig} \\
    300g & Weizenmehl 550       \\
    195g & Wasser               \\
    4g   & Salz                 \\
    1g   & Trockenhefe          \\
    \textbf{Hauptteig} \\
    & Vorteig              \\
    550g & Weizenmehl 550       \\
    415g & Wasser               \\
    18g  & Salz                 \\
    3g   & Trockenhefe          \\
    350g & Kartoffeln, geröstet
\end{tabular}\\\paragraph*{}
Für den Vorteig alle Zutaten miteinander vermischen und etwa 3-6 Minuten gut verkneten. Bei ca. 21$^\circ$C 12 Stunden gehen lassen. In der Zwischenzeit 350 g Kartoffeln mit Schale in Stücke von 1-2 cm Größe schneiden und im Ofen goldbraun und weich rösten, anschließend abkühlen lassen.\\
Alle Zutaten des Hauptteiges (bis auf den Vorteig) 3 Minuten verkneten und dann den Vorteig zugeben. Den Teig weitere 3-6 Minuten kneten und 1,5 Stunden bei 24$^\circ$C gehen lassen. Nach 45 Minuten 1-3 Mal den Teig falten. Den Teig in zwei Hälften teilen (bei Nutzung der Hälfte der Zutaten entfällt der Schritt natürlich), 10 Minuten abgedeckt entspannen lassen. Anschließend den Laib formen und für stabilen Halt wirken (meine Methode: Teigklumpen auf der bemehlten Arbeitsfläche zwischen beide Hände nehmen, gegen den Uhrzeigersinn drehen und dabei immer etwas Teig nach unten schieben). Nochmals 1,5 Stunden bei 24$^\circ$C gehen lassen und dann in den auf 230$^\circ$C vorgeheizten Backofen schieben. Anfangs eine halbe Tasse Wasser auf den Ofenboden schütten oder das Brot kräftig mit Wasser einsprühen. Nach 40-45 Minuten sollte das Brot dunkelbraun und fertig sein.
\newpage


\section{Brot mit ofengerösteten Kartoffeln - oder noch einfacher und super kartoffelig :)}
\begin{tabular}{ll}
    Zubereitung am Backtag:& 3 Stunden\\
\end{tabular}\\\paragraph*{}
\begin{tabular}{ll}
    \textbf{Hauptteig} \\
    680g  & Kartoffeln, mehlige       \\
    16g   & Salz                      \\
    120ml & Kartoffelkochwasser, warm \\
    1EL   & Trockenhefe               \\
    2EL   & Olivenöl                  \\
    665g  & Weizenmehl 550            \\
\end{tabular}\\\paragraph*{}
Die Kartoffeln vierteln und mit der Hälfte vom Salz kochen. Die Backhefe mit dem Kartoffelwasser vermischen und kurz quellen lassen (Dabei sollte das Wasser nicht mehr viel zu warm sein - Die Hefe stirbt ansonsten ab). Die abgekühlten Kartoffeln zerdrücken und mit den restlichen Zutaten verkneten. Der Teig ist am Anfang recht fest und krümelig, wird aber mit dem Kneten immer weicher und homogener.\\
Den Teig bei Raumtemperatur 20-30 Minuten gehen lassen. Anschließend zwei Brote abstechen und Laibe formen. Die Brote dann weitere 30 Minuten, mit Schluss nach unten, gehen lassen.\\
Das Brot dann 50 Minuten bei 190$^\circ$C mit Dampf backen. Nach etwa 20 Minuten Schwaden ablassen.

\newpage