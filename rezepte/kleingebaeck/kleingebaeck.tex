%! Author = jakob
%! Date = 18.03.22
%! kleingebäck

\chapter[Semmeln, Brezen und Co.]{Kleingebäck}


\section{Einfache Frühstückssemmel}
\begin{tabular}{ll}
    Zubereitung am Backtag: & $\approx$1,5 Stunden \\
    Zubereitung gesamt:     & $\approx$12 Stunden
\end{tabular}\\

\paragraph{}
\begin{tabular}{ll}
    \textbf{Hauptteig} \\
    25g  & Roggenmehl 997 (alternativ 1150)                          \\
    95g  & Dinkelmehl 630                                            \\
    330g & Weizenmehl 550                                            \\
    10g  & Salz                                                      \\
    14g  & Olivenöl                                                  \\
    300g & Wasser (15-18$^\circ$C)                                   \\
    4g   & Frischhefe                                                \\
    2,5g & Aktivmalzextrakt (aktives Flüssigmalz)(alternativ Zucker)
\end{tabular}\\

\paragraph{}
Alle Zutaten 3 Minuten auf niedrigster Stufe mischen und weitere 10-12 Minuten auf zweiter Stufe zu einem glatten, gut dehnbaren Teig kneten (Teigtemperatur ca. 25-26$^\circ$C). \\
Den Teig ca. 10 Stunden bei 7-8$^\circ$C im Kühlschrank lagern. \\
Am Morgen den Teig schonend auf die Arbeitsplatte geben und einmal vorsichtig dehnen und falten. \\
Nun ca. 9 quaderförmige Teiglinge abstechen und 1 Stunde in Leinen zur Gare bei Raumtemperatur stellen. \\
Tief einschneiden. Bei 250$^\circ$C fallend auf 230$^\circ$C mit viel Dampf 20 Minuten backen.\\\newpage


\section{Kieler Semmel}
\begin{tabular}{ll}
    Zubereitung am Backtag: & $\approx$3 Stunden     \\
    Zubereitung gesamt:     & $\approx$16-20 Stunden
\end{tabular}\\

\paragraph{}
\begin{tabular}{ll}
    \textbf{Vorteig} \\
    200g     & Weizenmehl 550                        \\
    150g     & Wasser                                \\
    1g       & Frischhefe                            \\
    \textbf{Hauptteig} \\
    & Vorteig                               \\
    125g     & Wasser                                \\
    5g       & Frischhefe                            \\
    15g      & Backmalz (oder 10g flüssiges Backmalz \\
    20g      & Butter                                \\
    10g      & Salz                                  \\
    optional & etwas Butter und Salz zum Scheuern    \\
\end{tabular}\\

\paragraph{}
Die Vorteigzutaten verrühren und 14-18 Stunden bei Raumtemperatur reifen lassen. Dann mit allen übrigen Zutaten 5 Minuten langsam und 10 Minuten auf 2. Stufe zu einem glatten, elastischen Teig kneten. 68 g-Teiglinge abstechen und rund schleifen. Auf der Arbeitsplatte einen dünnen Film mit weicher Butter streichen und etwas Salz darüber streuen. Die Teiglinge mit der glatten Seite nach unten im Fett-Salz-Gemisch kreisend bewegen bis kleine spiralförmige Schleifspuren auf dem Teigling erkennbar sind. Bei Bedarf die Arbeitsplatte erneut einfetten und salzen. Mit der gescheuerten Seite nach unten auf bemehltem Bäckerleinen 2 Stunden gehen lassen bis sich das Volumen verdoppelt hat. Die Teiglinge mit der gescheuerten Seite nach oben mit Wasser besprühen und bei 220$^\circ$C 20 Minuten backen. Anschließend nochmals ab-sprühen.\newpage


\section{Kaisersemmeln (handgeschlagen)}
\begin{tabular}{ll}
    Zubereitung am Backtag: & $\approx$3 Stunden  \\
    Zubereitung gesamt:     & $\approx$75 Stunden \\
\end{tabular}\\

\paragraph{}
\begin{tabular}{ll}
    \textbf{Vorteig (Pâte fermentée)} \\
    130g & Weizenmehl 550                \\
    85g  & Wasser                        \\
    4g   & Frischhefe                    \\
    3g   & Salz                          \\
    \textbf{Hauptteig} \\
    390g & Weizenmehl 550                \\
    35g  & Roggenmehl 1150               \\
    240g & Wasser ($\approx 20^\circ C$) \\
    4g   & Hefe                          \\
    7g   & Salz                          \\
    7g   & Butter                        \\
\end{tabular}\\

\paragraph{}
Die Vorteigzutaten vermengen und 3 Tage bei 3-4$^\circ$C reifen lassen.
Sämtliche Zutaten 5 Minuten auf niedrigster Stufe und 8 Minuten auf zweiter Stufe zu einem festen, straffen, leicht klebenden Teig verarbeiten (Teigtemperatur ca. 26$^\circ$C). \\
90 Minuten Gare bei 24$^\circ$C. Nach 30 und 60 Minuten falten.
9 Teiglinge zu ca. 100 g abstechen und rundschleifen. 10 Minuten ruhen lassen und anschließend in reichlich Mehl zu Teigfladen von ca. 10 cm Durchmesser flachdrücken. \\
Die Kaisersemmeln schlagen und mit Schluss nach unten 45 Minuten in Bäckerleinen bei 24$^\circ$C zur Gare stellen. \\
Mit Schluss nach oben auf ein Blech oder den Brotschieber setzen und mit Wasser abstreichen/absprühen. \\
Bei 230$^\circ$C 20 Minuten mit Dampf backen. Am Ende nochmals kräftig mit Wasser absprühen. \\\newpage


\section{Vegane Burger Buns}
\begin{tabular}{ll}
    Zubereitung am Backtag: & $\approx$2,5 Stunden   \\
    Zubereitung gesamt:     & $\approx$11-15 Stunden
\end{tabular}\\\paragraph*{}
\begin{tabular}{ll}
    \textbf{Hauptteig} \\
    400g & Weizenmehl 550                           \\
    85g  & Weizenvollkornmehl                       \\
    30g  & Roggen-Anstellgut                        \\
    7,5g & Frischhefe                               \\
    50g  & Apfel-Sanddorn-Mus (alternativ Apfelmus) \\
    25g  & Vollrohrzucker                           \\
    10g  & Salz                                     \\
    195g & Wasser (kalt)                            \\
    120g & Sonnenblumenöl                           \\
\end{tabular}\\\paragraph*{}
Alle Zutaten bis auf das Öl 5 Minuten auf niedrigster Stufe und weitere 5 Minuten auf zweiter Stufe kneten. Weitere 10 Minuten auf zweiter Stufe kneten und dabei das Öl langsam einlaufen lassen (Teigtemperatur ca. 26$^\circ$C).\\
Den Teig 8-12 Stunden bei 6-8$^\circ$C lagern.\\
100 g-Teiglinge abstechen, rundschleifen, in Sesam drücken und ca. 2 Stunden bei Raumtemperatur (ca. 20$^\circ$C) zugedeckt reifen lassen.\\
Bei 230$^\circ$C fallend auf 200$^\circ$C 20 Minuten mit viel Dampf backen.
\newpage


\section{Dinkelseelen}
\begin{tabular}{ll}
    Zubereitung am Backtag: & $\approx$30 Minuten    \\
    Zubereitung gesamt:     & $\approx$10-12 Stunden
\end{tabular}\\\paragraph*{}
\begin{tabular}{ll}
    \textbf{Mehlkochstück} \\
    12g  & Dinkelmehl 630         \\
    60g  & Wasser                 \\
    9g   & Salz                   \\
    \textbf{Hauptteig} \\
    & Mehlkochstück          \\
    375g & Dinkelmehl 630         \\
    25g  & Roggen-Anstellgut      \\
    180g & Wasser A (20$^\circ$C) \\
    120g & Wasser B (20$^\circ$C) \\
    8g   & Butter                 \\
    3g   & Frischhefe             \\
\end{tabular}\\\paragraph*{}
Die Mehlkochstückzutaten klümpchenfrei verrühren und unter Rühren aufkochen bis die Masse eindickt. Die Oberfläche mit Klarsichtfolie bedecken und alles auskühlen lassen (bis zu 24 Stunden bei Raumtemperatur lagerbar).\\
Alle Zutaten außer Wasser B 5 Minuten auf niedrigster Stufe und 1-2 Minuten auf zweiter Stufe zu einem glatten, dehnbaren Teig kneten. Dann nach und nach das Wasser B zufügen und einmischen. Der Teig ist sehr weich (Teigtemperatur ca. 24-25$^\circ$C).\\
Den Teig 8-10 Stunden bei 12-14$^\circ$C (oder 16-20 Stunden bei 5-6$^\circ$C) reifen lassen (Volumen sollte sich verdoppelt haben).\\
Den Teig auf die mit Wasser benetzte Arbeitsplatte geben, mit Kümmel und Salz bestreuen und ca. 100 g-Teiglinge mit den nassen Handkanten vom Teigrand abziehen.\\
Die Teiglinge auf Backpapier setzen und sofort bei 250$^\circ$C 15 Minuten mit Dampf backen.\\
\newpage


\section{Vorlage}
\begin{tabular}{ll}
    Zubereitung am Backtag: & $\approx$3 Stunden \\
    Zubereitung gesamt:     & $\approx$3 Stunden
\end{tabular}\\\paragraph*{}
\begin{tabular}{ll}
    \textbf{Tangzhong} \\
    20g  & Mehl, Type 550          \\
    27g  & Wasser                  \\
    60g  & Vollmilch               \\
    \textbf{Hauptteig} \\
    120g & Vollmilch, $30^\circ C$ \\
    9g   & Trockenhefe             \\
    320g & Mehl, Type 550          \\
    7g   & Salz                    \\
    35g  & Zucker                  \\
    1    & Ei                      \\
    1    & Eigelb                  \\
    42g  & Butter                  \\
\end{tabular}\\\paragraph*{}
Für den Tangzhong Mehl, Wasser und Vollmilch in einem Topf
vermsichen. Unter ständigem rühren auf dem Herd erhitzen, bis
die Flüssigkeit Puddingartig wird. Anschließend vom Herd nehmen
und abkühlen lassen.\\
Für den Teig die Milch, Hefe, Mehl, Salz, Zucker, Eier und Tangzong
vermischen. Mit einem Standmixer das Ganze 5 Minuten
kneten. Der Teig sollte jetzt homogen sein. Die Butter Stück für
Stück unter laufendem Mixer hinzugeben. Den Teig weitere 5 Minuten
kneten.\\
Den Teig glatt ziehen und in eine gefettete Schüssel legen. Die
Schüssel mit einem feuchten Handtuch abdecken und für 1-1,5h
gehen lassen oder bis sich der Teig in der Größe verdoppelt hat.
Anschließend den Teig ausstoßen und 6 gleichgroße Buns formen
(á 100g). Hierfür den Teig einfach wie normale Semmeln
rundschleifen, sodass die Teiglinge eine straffe Haut bekommen.
Die Buns für weitere 1-1,5 Stunden gehen lassen oder sie die
vollgare erreicht haben.\\
Für die Eierstreiche das Ei mit ein wenig Milch oder Wasser verrühren.
Die Buns einstreichen. Jetzt ließen sich auch gut Sesam
oder andere Körner auf die Oberfläche streuen.\\
Die Buns für 18 Minuten bei $200^\circ C$ backen. Wer mag, kann sie
anschließend mit geschmolzener Butter bestreichen, dadurch
bekommen sie einen schönen glanz.\\
\newpage